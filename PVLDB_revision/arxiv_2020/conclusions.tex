\enlargethispage{\baselineskip}

In this paper, we initiate the investigation of incentivized social advertising,
by formalizing the fundamental problem of revenue maximization from the
host perspective. In our formulation, incentives paid to the seed users are determined by
their demonstrated past influence in the topic of the specific ad. We show that,
keeping all important factors -- topical relevance of ads, their propensity for
social propagation, the topical influence of users, seed users' incentives, and advertiser
budgets -- in consideration, the problem of revenue maximization in incentivized
social advertising is NP-hard and it corresponds to the problem of monotone submodular
function maximization subject to a partition matroid constraint on the
ads-to-seeds allocation and multiple submodular knapsack constraints on the advertiser
budgets.
For this problem, we devise two natural greedy algorithms that differ in
their sensitivity to seed user incentive costs, provide
formal approximation guarantees, and achieve scalability by adapting to our context  recent advances made in scalable estimation of expected influence spread.

Our work takes an important first step toward enriching the framework of incentivized
social advertising with powerful ideas from viral marketing, while
making the latter more applicable to real-world online marketing. It opens up several interesting avenues for further research: \LL{$(i)$ it remains open whether our winning algorithm \fastcs can be made more memory efficient hence more scalable; $(ii)$ it remains open whether the approximation bound for \CSRM provided in Theorem~\ref{theo:cs-earm1} is tight; $(iii)$ it is interesting to integrate hard competition constraints into the influence propagation process; $(iv)$ it is worth studying our problem in an online adaptive setting where the partial results of the campaign can be taken into account while deciding the next moves.}
All  these directions  offer a
wealth of possibilities for future work.

%
%\LL{$(ii)$ the tightness of the approximation bound for \CSRM provided in Theorem~\ref{theo:cs-earm1} is open;\footnote{\small Interestingly, on the instance in the proof of Theorem~\ref{theo:CARM}, \CSRM obtains the optimal solution $T$.}} \LL{$(iii)$ it is worth studying our problem in an online adaptive setting where the partial results of the campaign can be taken into account while deciding the next moves.}

%Our work takes an important first step toward enriching the framework of incentivized
%social advertising with powerful ideas from viral marketing, while
%making the latter more applicable to real-world online marketing. It opens
%up several interesting avenues for further research: $(i)$ it remains open whether our winning algorithm \fastcs can be made more memory efficient and hence more scalable;   $(ii)$ capturing the auction dynamics
%of real-world social advertising by integrating algorithmic
%mechanism design techniques into the allocation of ads is an interesting problem; $(iii)$ it is interesting to integrate hard competition
%constraints into the influence propagation process; \LL{$(iv)$ it is worth studying our problem in an online adaptive setting where the partial results of the campaign can be taken into account while deciding the next moves.}
%All  these directions  offer a
%wealth of possibilities for future work.


